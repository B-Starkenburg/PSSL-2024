\documentclass[11pt, oneside]{article}   	% use "amsart" instead of "article" for AMSLaTeX format
\usepackage{geometry}                		% See geometry.pdf to learn the layout options. There are lots.
\geometry{a4paper}                   		% ... or a4paper or a5paper or ... 
%\geometry{landscape}                		% Activate for rotated page geometry
%\usepackage[parfill]{parskip}    		% Activate to begin paragraphs with an empty line rather than an indent
\usepackage{graphicx}				% Use pdf, png, jpg, or eps§ with pdflatex; use eps in DVI mode
								% TeX will automatically convert eps --> pdf in pdflatex		
\usepackage{amssymb}

%SetFonts

%SetFonts


\title{Draft: Beth Foundation Report for PSSL 109}
\author{C.~Ford}
%\date{}							% Activate to display a given date or no date

\begin{document}
\maketitle
%\section{}
%\subsection{}

		%[What was the purpose of your research project]					
The Peripatetic Seminar on Sheaves and Logic (PSSL) is a semi-regular,
informal meeting of category theorist, logicians, and type theorists. We had
three primary goals in mind for the 109th edition of PSSL:
\begin{enumerate}
\item
To increase visibility of Dutch research on categories and type theory;
\item
to provide an opportunity for young researchers (qualifier: MSc,~PhD,~postdoc) 
to expand their professional network and present their own research ideas;
\item
to facilitate the exchange of ideas across research and national borders.
\end{enumerate}

%[What were the most important activities?]			
%[Distinguished lectures]
The event took place over the weekend from 15-17th November with a
full program of 24 talks across 8 research themes. Our program included
presentations from researchers based in at least four Dutch institutes (UvA, 
LEI, UG, and UU), 17 talks from young researchers (cf. item 2 above), and
participants from at least 11 countries (both in-person and virtual). We were 
satisfied with these results as they contributed towards Primary Goal 1, 
Primary Goal 2, and Primary Goal 3, respectively. 

Moreover, we allocated a portion of our budget towards a program of two 
distinguished/invited lectures. Our distinguished speakers were Maaike Zwart 
(IT University of Copenhagen) and Paolo Perrone (University of Oxford): both 
are outstanding early career researchers in accordance with Primary Goal 2. 
They contributed tutorial-style lectures on distributive laws of monads and 
categorical probability theory, respectively.

Another important aspect of our event was the conference dinner. This 
provided a space for more informal discussion and an opportunity for the
community to get to know one another on a personal level. The impact of
this activity further facilitated Primary Goal 1 and Primary Goal 2.		
\end{document}  